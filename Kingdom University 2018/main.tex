\documentclass[a4paper]{article}

\usepackage[english]{babel}
\usepackage[utf8]{inputenc}
\usepackage{amsmath}
\usepackage{graphicx}
\usepackage[colorinlistoftodos]{todonotes}
\usepackage{url}
\usepackage{blindtext}
\usepackage{verbatim}
\title{As American as apple pie}

\author{Yda van Boulogne}

\date{\today}

\begin{document}
\maketitle

\begin{abstract}
Apple pie in the low countries in the 15th - 17th century.
\end{abstract}

\section{Introduction}
\label{sec:introduction}
In KANTL 15, a 15th century cookbook from the Southern Netherlands there are two recipes for taerten or tarte (tarte). No further qualifications are necessary. Both of these would qualify as apple pie nowadays. Apple pie is a common fixture amongst recipes for baked goods. Appearing in forms recognisable as almost modern apple pie, meat replacement and being accompanied by a wide variety of spices and other ingredients. 

The most important ingredient in apple pie is in general the apple. There are older varieties available, possibly in your local supermarket this time of year. Whether or not it’s worth locating these varieties is up to you. A wide variety of apples was available in period, both in size and in flavour and the exact type of apple tree standing next to the house was conceivable specific to that particular tree as apple trees grown from seed seldom breed true. An short overview of medieval apple varieties can be found on \url{http://medievalcookery.com/notes/fruit/fruit.html} . I’ve had good results with Jonagolds, Granny Smiths  and interesting results with Goudrenette. Modern eating apples like a Jazz have been less successful. Pick your apple with your crust in mind, drier apples can become quite dry in an open tarte, a juicier apple is better in that case. If you’re making a double crusted pie you’re more likely to endup with applesauce in the pie anyways but you might want to avoid very juicy apples. 

Apple pies are not just found in the cookbooks from the low countries. Just like modern apple pie exists in many forms depending on the country period apple pie recipes appear at least in English, German and Italian cookbooks. Some varieties are even known by regional names.

The oldest recipes for I’m aware of for apple pie are from the English corpus, A forme of Currye has a 14th century apple pie recipe (A redaction can be found on \url{http://www.godecookery.com/twotarts/twotarts.html}). And a quite similar recipe shows up in the cookbooks from the low countries as well. 



\section{Available sources}
\label{sec:avSources}
\begin{description}

\item[UB 1035, Wel ende Edelicke Spijse] A manuscript from the 2nd half of the 15th century. Transcriptions and English translations copied from \url{https://coquinaria.nl/kooktekst/Edelikespijse0.htm} by Christianne Muusers

\item[KANTL 15] A late 15th, early 16th Century manuscript from the southern low countries.  Transcriptions and English translations copied from \url{https://coquinaria.nl/kooktekst/KA15Gent02.htm} by Christianne Muusers

\item[Een notabel boecxken van cokeryen] Printed in Brussels ca 1514. Transcriptions for this cookbook are copied from \url{http://users.telenet.be/willy.vancammeren/NBC/index.htm}, Ria Jansen-Sieben and Marleen van der Molen-Willebrands
\item[UB 476] A manuscript from the 16th century. Reprinted as De keuken van de late middeleeuwen by Ria Janssen-Sieben en Johanna Maria van Winter 
\item[Eenen nyeuwen coock boeck] Printed in Antwerp in 1560, reprinted in 1971 as “Eenen nyeuwen coock boeck. Kookboek samengesteld door Gheeraert Vorselman en gedrukt te Antwerpen in 1560” by E. Cockx-Indestege, 
\item[Cocboeck] Printed in Dordrecht in 1593. Transcriptions for this cookbook are copied from \url{http://www.kookhistorie.nl/cb/index_cb.htm}, copyright  Marleen Willebrands. 
\item[Koocboec oft familieren keukenboec] Printed in 1612 in Leuven, based on Scapi. Transcriptions for this cookbook are copied from , Jozef Schildermans, Hilde Sels en Marleen Willebrands. Reprinted as  “Lieve schat, wat vind je lekker? Het Koocboec van Antonius Magirus (1612) en de Italiaanse keuken van de renaissance.”
\item[De verstandige kock of sorghvuldige huyshoudster] A printed cookbook with editions from 1667 and 1669. Printed in Amsterdam. Transcriptions for this cookbook are copied from \url{http://www.kookhistorie.nl/VK/index_vk.htm} copyright Marleen Willebrands
\item[De hoofsche pasteybacker] A subsection in ‘De verstandige kock’ 1669. Transcriptions for this cookbook are copied from \url{http://www.kookhistorie.nl/HP/index_hp.htm}, copyright  Marleen Willebrands

\end{description}

Een notabelen boecxken van cokeryen ->  Cocboek
MS476 -> Cocboek
Eenen nyeuwen coock boeck -> Cocboek
Cocboek -> De verstandige kock

\section{Cooking techniques}
Different techniques for cooking the filling are used:
\begin{itemize}
    \item chopped small and immediatly added to the dough
    \item cooked whole next to the fire and peeled afterwards
    \item panfried prior to being added to the dough
    \item cooked in the dough and put back in the oven with additional ingredients
    \item stewed in wine or butter
\end{itemize}

Depending on the recipe a piepan may be mentioned or not. In some cases there is explicit mention of being a doublecrousted pie and in other sources there is explicit mention of being a single crusted pie. Ussually there is no mention apart from put it in dough. The pie's are baked in piepans (which allows for top and bottom heat by way of putting coals on the lid of the pan and beneath) or in an oven.

\section{Pie crust} 
Several different types of crust show up, variations of hot water crust, short crust pastry enriched with egg white as well as other enriched doughs. Doughs could be colored and/or spiced as needed for instance by including adding rosewater or saffron to the dough or washing the dough with milk or egg before baking. Lard is not featured in preparing crusts instead butter and canola oil are used (oil from rapes).


\section{Some recipes}






\subsection{Recognisable things}
Just like many current apple pies these recipes use spices like cinnamon, ginger, cloves, nutmeg and occasionally pepper. While ginger can seem odd in sweet dishes initially it’s still an ingredient in speculaas (biscoff) spices, a mixture also known modernly as koekkruiden in the Netherlands, and pumpkin spice. Varying the amount of ginger and peper in these recipes will give you very modern flavoured things or something just outside of it.
Apart from apples other common ingredients are figs and raisins, easy sources of sweetness when sugar is still expensive and not necessarily available, and nuts such as pine nuts or some almond flour to absorb the juices of the apples.

\subsubsection{figgs, raisins and spices}
\medskip
\begin{minipage}{.45\textwidth}
[55]  Om taerten in den vasten 
Neemt vijghen ende rosijnen gember peeper naegelen suijcker appelen gestooten in eenen mortier ende gebacken in den ouen	\cite{KANTL15}
\end{minipage}
\begin{minipage}{0.05\textwidth}
\ \ \ 
\end{minipage}
\begin{minipage}{.45\textwidth}
2.55. For a cake during Lent
Take figs, raisins, ginger, pepper, cloves, sugar and apples, ground in a mortar and baked in the oven. \cite{KANTL15}
\end{minipage}



\medskip
\begin{minipage}{.45\textwidth}
[6]. Om een tarte te maeken in die vasten Neemt appelen caneel suycker ende vygen ende rosynen ende gengeber \cite{KANTL15}	
\end{minipage}
\begin{minipage}{0.05\textwidth}
\ \ \ 
\end{minipage}
\begin{minipage}{.45\textwidth}
2.6. To make a pie for Lent
Take apples, cinnamon, sugar, figs, raisins, and ginger. \cite{KANTL15}
\end{minipage}
\ \ \ 
\paragraph{Redaction:}
Good for 1 pie in an Ikea large pie pan:

\paragraph{Ingredients}
\begin{itemize}
\item 1 Piecrust, I made a hot water crust with 500g flour (2/3 go on the bottom, 1/3 on the lid)
\item 7 Apples, Granny Smith
\item 40g sugar
\item 1 tsp cinnamon
\item 1/2 tsp ginger
\item handful of raisins
\item 5 dried figs quartered
\end{itemize}

\paragraph{instructions}
\begin{enumerate}
\item Peel apples, quarter and core (if not adding a lid, chop into small pieces)
\item Mix in a small bowl sugar, cinnamon and ginger
\item Make your pie crust if needed
\item Spread figs and raisins on the bottom of the pie crust
\item Lay your apples on the pie crust
\item Sprinkle with the sugar and spice mixture
\item Add lid to pie, making sure to close the edges by pinching the two layers of dough together
\item Serve as lunch
\end{enumerate}






\subsection{Slightly less recognisable things: custard pies}
Several versions of apple pie incorporate dairy (cream or cheese) and eggs. These are often (but not always) called 'vladen'. A word related to vlaai, a specific kind of type of modern fruitpie from the Limburg region or to vla, a type of custard sold in tetrapacks in the modern Netherlands.\cite{PhilippaEtymologisch}. The recipes in this section are not necessarily baked in a pieshell in at least one recipe the instructions are to put it in a dish and cook till it no longer attaches to the dish.

Vladen is also a larger category of baked goods where eggs get combined with fruit and or cream. 

\subsubsection{eggs and cheese}
\medskip
\begin{minipage}{.45\textwidth}
149. Toerte van appelen.
Laet de appelen wat onder heete asschen braden, doet hun dan de verbrande schellen aff, maecktse schoon naer behoorte, stamptse in eenen mortier, met twee oncen mostacciolen, tot elcke twee pont appelen, ende vier oncen gheraspten kese, ende ses oncen recotten, ende als dit al sal gestooten wesen, doetter by ses doyeren van rouw eyeren, ende noch twee met het wit, een loot canneel, een loot tusschen peper, nagelen, ende noten, acht oncen suycker: maeckt hier u toerte aff. Wilt ghy de appelen niet laten braden, soo moet mense wat laten stoven in de boter, synde eersten in stuxkens gesneden, ooc mach mense stoven met wat witten wijn, suycker, ende roosewater. Dese toerte can te passe comen in alle saisoen, ende men dientse warm oft kout.
\end{minipage}
\begin{minipage}{0.05\textwidth}
\ \ \ 
\end{minipage}
\begin{minipage}{.45\textwidth}
149 Pie of apples
Cook the appels under hot ashes. Take of the burned peels and clean them well. Pound them in a mortar with two ounces sugar for every 2 pound of apples and 4 ounces grated cheese and six ounces ricotta, and when this has been pounded add 6 egg yolks and two with the white, a loot (+- 15g) cinnamon, a loot pepper, 
\end{minipage}


\subsection{Slightly less recognisable things: Fennel}
\medskip


\medskip
\begin{minipage}{.45\textwidth}
KANTL 15
[77] Om een venckeltaert
Neempt plat dun breet broot ende snyt die appel al dunne ende onderlegst al met kaneel zuycker ende een lutken bloem van muscaten ende een lutken naegel ende legt al breet broot ende vol boteren ende dan strauter venckel zaet op ende dan sluijtet toe met eenen dunnen decsel ende bact ende eetet al werm des gelycks tuschen twee scotelen dan blyft daer het broot wijt	
\end{minipage}
\begin{minipage}{0.05\textwidth}
\ \ \ 
\end{minipage}
\begin{minipage}{.45\textwidth}
2.77. To make a fennel pie
Take flat, thin and broad bread. Cut the apples very thinly, and temper them with cinnamon, sugar, a little mace and a little cloves. Put them on all the bread, covered with butter, and then sprinkle fennel seeds on it and cover with a thin lid. Bake it, and eat all hot. The same with between two dishes, the bread will then remain white.
\end{minipage}
\ \ \ 
\paragraph{Redaction}
Good for 1 large Ikea piepan, single crusted

\paragraph{Ingredients}
\begin{itemize}
\item 1 pie crust
\item 500g apples, peeled and cored, chopped small
\item 1 ts cinnamon
\item 1/2 ts nutmeg
\item 2-6 tbs sugar (depends on the sweetness of your apples)
\item 1-2 tbs fennel seeds
\item 1 tbs butter
\end{itemize}

\paragraph{Instructions}
\begin{enumerate}
\item Make a piecrust, whichever one you like
\item Mix cinnamon, nutmeg and sugar in a separate bowl
\item Mix apples with sugar and spice mixture
\item Fill piecrust with apples
\item divide butter in small pieces and spread on top of filling
\item Sprinkle fennel seeds on top
\item Bake
\end{enumerate}







\subsection{You put what in there?}
Amongst the more esoteric ingredients we find rose water (beware not everybody likes it and reminds people of soap), parsley (haven’t succeeded in creating something edible that also had parsley flavor yet)




\subsection{Vegetarian alternatives (aka apple pie which looks pretty)}
Probably the most unexpected apple pie is apple pie shaped like meat. Period food was often shaped to look like something it wasn’t and in this case there are a few recipes for fake fish (KANTL15 and Eenen nyeuwen coock boeck) or fake calfs-ears (UB 476) meant to be eaten in lent.

\medskip
\begin{minipage}{.45\textwidth}
Om gheuormde wijs te maken in die wasten ende oeck calfsoeren. (KANTL15)
stoet jn enen mortijer vijf of sees appellen schon gheschelt sonder kersel huijs ende doter jnne van ghestoten amandellen of gheroost pepercock met een luttel sofferaens ende backt dese jn olye of mackt groten wijs backse gheuerwet ende van gheghat jnden ouen	To make formed fish during lent and also calf ears.
\end{minipage}
\begin{minipage}{0.05\textwidth}
\ \ \ 
\end{minipage}
\begin{minipage}{.45\textwidth}
Crush in a mortar five or six apples, peeled and cored. Add sugar, ginger and cinnamon, and add some pound almonds or toasted gingerbread with some saffron. Bake this in oil. Or make a big fish: bake this in the oven, painted and with some holes in it.
\end{minipage}

\medskip
\begin{minipage}{.45\textwidth}
Item calfs oeren maeckt aldus (UB476)
nempt gheplet deck sausijer ronde maeckt dat dobbel ende dan slaet die tve langen eynden te samen ende dan nempt scherp eynde tussen tve wijnhgheren ende steckt jrst dat runt ende en luttel daer nae met allen ende als dit stijf is nempt dat wijt ende doet daer jnne vanden vorseyde stof sonder sieden ende dijnt dat.
\end{minipage}
\begin{minipage}{0.05\textwidth}
\ \ \ 
\end{minipage}
\begin{minipage}{.45\textwidth}
	Calf ears are made thus: Take the flattened dough, rounded like a saucer. Make it double, and take the two long ends together. Then take the pointed end between two fingers, and put first the rounded end in [the boiling oil], and shortly afterwards the whole. Take it out when it is crunchy, and put some of the afore mentioned stuffing in it without boiling [it], and serve it.
\end{minipage}


\section{Collected other recipes}


\medskip
\begin{minipage}{.45\textwidth}
Om appel taerten:
p.55 Neemt appel al reyn gepelt ende snytse in vier oft ach stucken ontwee ende legtse int broot ende sluijtse ende schietse in den ouen ende alst gebacken is neemptse weeder uuyt dan brysse ontwee ende neempt wat eys doeyer ende wat peepercoeck ende slaetet duer een zye ende cruijt met naegel caneel lutken gember ende greyn ende een lutken galigaen ende dan in een teyl op claer colen of in den ouen ende ij oft iij weruen geroert ende sofferaen daer in ende dan weeder int broot alst ghyt vuijten broode doet maect een cleyn gaetken gietet daerin ende als die spys alsoe gemaect is dan doetse daer weder in ende legter stucken op	\cite{KANTL15}
\end{minipage}
\begin{minipage}{0.05\textwidth}
\ \ \ 
\end{minipage}
\begin{minipage}{.45\textwidth}
2.76. [To make] apple pie
Take clean[ed] and peeled apples, cut them in four our eight parts, end put them in the bread. Close them and put them in the oven. And when they're baked, take them out again and bray them finely. Take some egg yolks and some gingerbread and strain it [with the pureed apples] and spice it with clove, cinamon, a little ginger and grains of paradise and a little galangal. Then put it in an earthenware dish on clear coals, stir 2 or 3 times, add saffron and then [put it] back in the bread if you want to serve it out of bread. Make a small hole and pour it in, and when the stuffing is thusly made, put it back in and cover with pieces [of bread]. \cite{KANTL15}
\cite{KANTL15}
\end{minipage}

\medskip
\begin{minipage}{.45\textwidth}	
30 Om een appeltourt te maecken.
Set een korst in de taertpan op die wijse, als hier voor geseght is. 
Strooy een laegh suycker op de bodem, en stoffeer hem daer nae met appelen, 
of peeren, die geschilt, en kleyn gehackt, of aen daelders gesneeden sijn: 
doch met moet de pitten en klockhuys daer uyt neemen. Mengh’er pingelen, 
korenten en limoenschil onder, indien ghy ’t soo begeert. 
Bestrooy dit alles met een weynigh gestoote kaneel, voegh’er poeyersuycker naer 
uw believen by; gelijck oock een brock varsche butter, soo groot als een groote 
noot voor een middelmatighe tourte. Deck dan dese tourte met een decksel, 
dat deurgesneden en gekoleurt is, en set haer in d’oven. Bestrooy haer met suycker, 
als sy gaer is, en set haer weer voor een korte tijt in. \cite{hp}
\end{minipage}
\begin{minipage}{0.05\textwidth}
\ \ \ 
\end{minipage}
\begin{minipage}{.45\textwidth}
30 To make an apple pie
Make a crust in the pie pan in the way mentioned above. Sprinkle sugar on the bottom of the crust and place apples or pear on it, peeled and cored, diced or sliced. Add pine nuts, raisins, and lime peel if you so desire.
Sprinkle this with some powdered cinnamon and powdered sugar as well as a piece of fresh butter the size of a large nut for a medium pie. Close the pie with a lid which has been cut and colored and put it in the oven. Sprinkle it with sugar once it is done and put it back for a short while. 
\end{minipage}

\medskip
\begin{minipage}{.45\textwidth}
128 Ander manier om een appeltaert te backen 
Neemt een pont meel, 4 lepelen water met 2 of 3 lepelen suycker (dit opgekoockt als een syroop). Doet daerby een ey, voorts sooveel boter totdat mender een bequame deegh van kan maken. Dit geeft een goede korste. Voorts neemt men 20 of 25 goede appelen na die groot zijn, geschilt en kleyngesneden. Doet die in een aerden pot, met boter daerby op 't vuur geset en dickmael omgeschut tot die gesloncken zijn. Dan die in u korst gedaen tot de boom bedeckt is, dan weer suycker, korenten en boter, dan weer appelen, tot u korst vol is, en dan op 't vuur gedaen. \cite{vk}
\end{minipage}
\begin{minipage}{0.05\textwidth}
\ \ \ 
\end{minipage}
\begin{minipage}{.45\textwidth}
\end{minipage}

\medskip
\begin{minipage}{.45\textwidth}
129 Om een appeltaert te maken op noch een ander wijse
Neemt appelen, schiltse en snijtse aen vierendeelen en doet er de klockhuysen uyt, en dan noch aen kleynder dunne snippers gesneden, drie vierendeel korenten schoongewasschen, en drie vierendeel suycker, een loot gestoten caneel. Dan den deegh in de pan geleyt en eerst appelen daerin gestroyt, dan korenten, suycker en kaneel, en stucken boter, en dit soo al met lagen geleyt tot de pan vol is. Sommige doender oock wel gestoten annijszaet in. Dan een decksel van deeg daerbovenop, en hier en daer een gat in het decksel gesneden, en soo met vuur onder en boven laten backen.	 \cite{vk}
\end{minipage}
\begin{minipage}{0.05\textwidth}
\ \ \ 
\end{minipage}
\begin{minipage}{.45\textwidth}
translation
\end{minipage}	






 
 
 \medskip
\begin{minipage}{.45\textwidth}
{5 Om koleur aen de pasteykorst te geven}
Klop het wit en dojers van eyeren te samen, als of ghy een struijf daer af begeerde te maecken. En indien ghy wilt dat de korst hoogh van koleur sal zijn, so sal ’t genoegh zijn ’t wit van een ey alleen met twee of drie dojers te menghen. Maer indien men de korst bleeck van koleur begeert, soo sal men slechs de dojer der eyeren nemen, en de zelfden met een weinigh water mengen en deurkloppen.
Om dit mengsel te gebruijcken, so neem een sacht quastje van veeren, of van een borstel, op dat de korst niet geschuert word. 
Indien ghy geen eyeren tot dese bestrijckingh wilt besighen, soo kont ghy een weynigh safferaen, of goutsboem in melck doen, of honigh in plaats van d’eyeren gebruijcken, gelijck de pasteybackers, om d’eyeren te sparen. \cite{hp}
\end{minipage}
\begin{minipage}{0.05\textwidth}
\ \ \ 
\end{minipage}
\begin{minipage}{.45\textwidth}
\end{minipage}

\subsubsection{Hot Water Crust pastry}
For hot water crust pastry you heat water and fat and add flour to the hot water

\medskip
\begin{minipage}{.45\textwidth}
[89a] het deech maect men tottten taerten of vladen: KANTL15 (15th/16th C)
men schuddet meel in ziedende water ende wat gesmolten booter daerin	
\end{minipage}
\begin{minipage}{0.05\textwidth}
\ \ \ 
\end{minipage}
\begin{minipage}{.45\textwidth}
2.89a The dough one makes for pie: 

one sifts meal into boiling water and adds some butter.
\end{minipage}

\subsubsection{Short Crust Pastry}
Short crust pastry involves rubbing fat into flour and after adding a small amount of liquid to create a cohesive dough.


\medskip
\begin{minipage}{.45\textwidth}
[78] fyn broot: 
om fyn broot hier al toe te maeken neempt terwenbloemen ende wat raepsmout niet vele ende wit van eyeren \cite{KANTL15}
\end{minipage}
\begin{minipage}{0.05\textwidth}
\ \ \ 
\end{minipage}
\begin{minipage}{.45\textwidth}
	To make fine bread for this: take wheat flour and some rape oil, not much, and egg white. \cite{KANTL15}
\end{minipage}


\medskip
\begin{minipage}{.45\textwidth}
226 Om een excelente taerte te backen van seer cort deegh.
Neemt fijne tarwenblommen in een schotel ende doet er een rau ey in, ongeclopt of is de taerte groot, so neemt noch eenen rauwen doyer daerby ende breetse wel in de oft met de bloeme ontstucken. Neemt dan een vyerken waters, deylt dat in dryen, neemt het een deel ende doetet in een commeken. Doet daerin een clomp boters, so groot of wat meer als een half ey. Set dit op een cafoor met vier ende latet tsamen so warm werden, dat de boter daer maer recht in en smelte. Maect dan voorts u deegh daermede. Wercket deegh wel tesamen, neemt dan de een helft van u deegh ende houtse in gelijcke warmte int commeken ende rolt de ander helft seer dunnekens uut ende vouwet dan in vieren. Stroyt wat bloemen in u toertpanne, oft stroyt boter onder in de panne die niet gewent te rijden en is, in stede van de bloemen ende legt dit deegh in de panne ende ontfouwet so in de panne opdat u deech niet en breke ende laet het deegh over de canten van de panne hangen totdat ghy er het scheel op legt. Sporet dan tsamen af, rolt terstont het ander deegh seer dunne uut ende sporet keperwijs oft sulcke fatsoen als ghy begeert. Vouwet dan in vieren ende latet lochtgens ligghen totdat ghy u taerte gevolt hebt. <…>  Legt er dan u gespoort scheel op ende ontfouwet alleynskens dattet niet en breke. Sporet deegh dan aen de canten tegen de panne af, maer douwt het deegh eerst met den vynger aen malcanderen ende neemt een quispelken, overstrijckt u taerte met gesmolten boter. Dect u panne ende setse op weynich vyers, maer legt bovenop het scheel noch sovele vyers als onder ende besietse dickmaels datse niet en brande, ende alsse bycans gebacken is, so verschutse altemet datse onder niet aen en brande. Alsse gebacken is, so stroyt er suycker ende wat caneel over ende dientse. Van dit deegh meuchdy alle taerten, pasteyen ende toerten backen. \cite{cb}
\end{minipage}
\begin{minipage}{0.05\textwidth}
\ \ \ 
\end{minipage}
\begin{minipage}{.45\textwidth}
	To make an excellent pie of very short dough
Take fine wheat flour in a dish and add a raw egg, unbeaten or if the pie is large take an extra raw egg yolk and mix it well with the flour. Take a [vyerken] water, divide it in three, take a part and put it in a bowl. Add therein some butter the size or a bit more of half an egg. Put this on the firepan with [coals] and let it become warm, that the butter melts. Than make your dough with it. Work the dough well together, take half of the dough and hold it in equal heat in a bowl and roll the other half very thinly and fold it in four. Sprinkle some flour in your piedish or sprinkle butter under in the pan which is not used to ride and is instead of the flour and put this dough in the pan and unfold it so in the pan so your dough doesn't break. and let the dough hang over the sides of the pan until you put the lid on it.
\end{minipage}

\subsubsection{Enriched doughs}
Enriched doughs contain additional fat, sugar or eggs. I’ve classed the doughs with just egghwite under the shortcrust section and i’m keeping the ones with significantly increased amounts of fat, sugar and/or eggs in this section.


\medskip
\begin{minipage}{.45\textwidth}
4 Om een suyker pasteykorst te maecken. 
Neem tot een voorbeelt een vierendeel poejersuyker, deur een zift gedaen, en doe de selfde in een suyvere marme[p.178]re vijsel. Doe’er ’t vierendeel van ’t wit van een ey by, en omtrent een halve lepel vol van citroensap. Roer dese dingen soetelijck te samen tot dat de suyker begint te smelten. En indien de selfde swarelijck wil smelten, soo doet men eenige druppelen van rooswater daar by. Voorts, als de suycker begint dick te worden, sal men met de stamper daer op stoten, om een vast deegh daer af te maecken, en, als sy wel gestremt en gemengt is, de korst van de pastey daer af toestellen.
    Men maeckt somtijts oock korsten half van suycker met even veel suycker en tarwe meel te mengen, en op de selfde wijse, als nu vertoont is, toe te maken.	\cite{hp}
\end{minipage}
\begin{minipage}{0.05\textwidth}
\ \ \ 
\end{minipage}
\begin{minipage}{.45\textwidth}
\end{minipage}


 

\medskip
\begin{minipage}{.45\textwidth}
32 Om te maken een taerte die men noemt Dornijpe taerte.
Neemt appelen cleyngecapt ende doet erby doyeren van eyeren, caneel ende suycker soo veel u goet dunct ende gesmolten boter. Menget dit ondereen ende legghet in u deegh ende latet backen.	\cite{cb}
\end{minipage}
\begin{minipage}{0.05\textwidth}
\ \ \ 
\end{minipage}
\begin{minipage}{.45\textwidth}
32 To make a pie which one calls pie from Doornik
Take apples, chopped small and add yoks of eggs, cinnamon and sugar as much as you think it needs and melted butter. Mix this and put it in your dough and let it bake
\end{minipage}



\medskip
\begin{minipage}{.45\textwidth}
46 Om appelvladen te maken.
Neemt guldelingen, scheltse ende snijtse in stucken ende doetse in eenen pot met wat wijns ende boter ende laetse so staen smooren. Wrijftse wel cleyn ontwee ende doet er dan by half soo veel gheraspt wittebroot als ghy appelen hebt ende vijf doyeren van eyeren, gengeber ende suycker. Mengelt dit al tesamen ondereen. Dit is tot twee schotelen. Bestrijckt u schotelen met boter ende doet er dan u spijse in ende set u schotelen op coolvyer ende latet backen tot dattet so stijf is dattet van de canten lichtet. Stroyt er dan suycker ende canneel op ende dienet ter tafelen.	\cite{cb}
\end{minipage}
\begin{minipage}{0.05\textwidth}
\ \ \ 
\end{minipage}
\begin{minipage}{.45\textwidth}
46 To make applevladen
Take Guldelingen [a type of apple], peel them and cut them in pieces and put them in a pot with wine and butter and let them braise. Rub them small and add half as much grated white bread as you have apples and five yoks of eggs, ginger and sugar. Mix this all together this is for two dishes. Baste your dish with buter and add your [spijs] in end put your dish on coalfire and let it bake till it is so stiff it loosens from the sides. Sprinkle with sugar and cinnamon and serve it to the table. 
\end{minipage}

\medskip
\begin{minipage}{.45\textwidth}
225 Om taerten van appelen te maken op de Walsche maniere.
Neemt appelen geschelt ende snijtse in quartieren. Sietse wel morwe in Rijnschen wijn, boter, suycker, gengeber, corenten. Als het tsamen wel morwe gesoden is, roert er dan twee doyeren van eyeren in. Legt u spijse in fijn deegh ende bactse als boven ende het is gereet.	Also found in De verstandige kok 130 \cite{cb} Also found in 'De Verstandige Kok' as recipe 130 

\end{minipage}
\begin{minipage}{0.05\textwidth}
\ \ \ 
\end{minipage}
\begin{minipage}{.45\textwidth}
225 To make a pie of apples in the Wallon way

Take apples and peel them and cut them in quarters. Boil them in Rhine wine with butter, sugar, ginger, and currants till they are soft. Once it is boiled soft together stir in two egg yolks. Put your filling in a fine dough and bake as above and it is done.

The recipe above is a recipe for a dry cake (koek) baked in a ‘toertpan’. While the reference to the dry cake seems a bit complicated it might be referring to the technique used for baking. It’s baking the cake in a ‘toertpan’ which you put on a raster with fire above and below. This sounds like a predecessor to a Dutch oven.
\end{minipage}

Ingredients:

a piecrust of your choice
apples, cooking apples preferred, they don’t have to hold their shape
Rhine wine, e.g. Rieslinger
butter
sugar
powdered ginger
currants
Peel the apples and chop them is pieces. Boil the apples with the butter, sugar, currants, and ginger in the whine till soft.
Stir in some egg yolks (take two for a large Ikea piepan)
Pour filling into pie crust
Bake as directed for the crust.

\medskip
\begin{minipage}{.45\textwidth}
261 Een appelvlaeye te maken.
261a Neemt eenen ketel appelen, anderhalf pepercoec, twee potten waters, gengeber, peper, canneel, nagelen, ende so menigen pot melcx als gy er in doet, so doet er so dicmaels veerthien eyeren in ende so menigen pot waters, sovele noten, syrope ende eyeren. \cite{cb}

\end{minipage}
\begin{minipage}{0.05\textwidth}
\ \ \ 
\end{minipage}
\begin{minipage}{.45\textwidth}
261 To make an appelvlaeye
Take a kettle of apples. a gingerbread (a spicebread from the low countries), two pots water, ginger, pepper, cinnamon, cloves and as much milk as you will add. So add some 14 eggs in en as many pots of water, many nuts, syrup and eggs.
\end{minipage}

\medskip
\begin{minipage}{.45\textwidth}
261b Op een ander maniere. 
Neemt appelen, sietse met pepercoec ende alsse morwe zijn, neemt peper end nagelen, foeylie, gengeber, syrope, suycker, veel eyeren ende boter.	\cite{cb}
\end{minipage}
\begin{minipage}{0.05\textwidth}
\ \ \ 
\end{minipage}
\begin{minipage}{.45\textwidth}
261b In another way
Take apples, boil them with gingerbread and when they are soft take peper and cloves, mace, ginger, syrup, sugar, many eggs and butter
\end{minipage}


\begin{minipage}{.45\textwidth}
135 Om een doornepen appeltaert te maken
Neemt appelen, kleyngekapt, en doet erby doren van eyeren, kaneel en suycker, en gesmolten boter. Menght dit ondereen, leght het in deegh en laet het backen.\cite{vk}
\end{minipage}
\begin{minipage}{0.05\textwidth}
\ \ \ 
\end{minipage}
\begin{minipage}{.45\textwidth}
\end{minipage}

\medskip
\begin{minipage}{.45\textwidth}
137 Om venkeltaerten te maken (Keuken van de late middeleeuwen) 
neempt appelen en snijtse [snijd ze] dunne ende neempt venkel ende booter, backtse [bak ze] daer mede; als sy gebacken is, dan neempt caneel ende suker ende boter, ende laetse [laat ze] noch sooken [sudderen].	\cite{KLM}
\end{minipage}
\begin{minipage}{0.05\textwidth}
\ \ \ 
\end{minipage}
\begin{minipage}{.45\textwidth}
To make fennelpie
Take appels and [slice them] thinly and take fennel and butter. Bake [them] with it. When she is backed take cinnamon and sugar and butter and let it stew\cite{KLM}
\end{minipage}

\medskip
\begin{minipage}{.45\textwidth}
 [87] om en appeltart te maken of een wenkl tart te maken KANTL 15
die wenkel tart snijt die appellen tot schijven ende die appeltart salmen die appel cappen jn elck tart salmen doen xxxv of xl appel nae dat sij groet sijn ende jn elck tart j loet canels j loet ghenbars j virdelloet pepers ende suars nae den mont 	\cite{KANTL15}
\end{minipage}
\begin{minipage}{0.05\textwidth}
\ \ \ 
\end{minipage}
\begin{minipage}{.45\textwidth}
1.87. To make apple pie or fennel pie
[To make] fennell pie cut the apples in slices, [to make] apple pie one must chop the apples. In each cake one shall use 35 or 40 apples according to their size, and in each cake [put] 1 lead cinnamon, 1 lead ginger, 1 quarter lead pepper, and sugar to taste \cite{KANTL15}
\end{minipage}

\medskip
\begin{minipage}{.45\textwidth}
228 Om een appeltaerte te maken.
Coc-boeck 1593
Maket deegh als voorsc. is. Capt u appelen cleyn oft bycans cleyn. Doetse dan in een schotel ende stroyt er wel suycker ende canneelpoeder op met wat gengeber ende wat roosewater. Mengelet wel tsamen onder de appelen. Legt dese spijse in u deech ende in u panne. Steect hier ende daer in tusschen u spijse een stucxken versche boter, legt u gespoort scheel daerop als voren ende backet als een spenagetaerte. Alsse nu genoech ghebacken is, stroyt er suycker ende canneel over. Ghy meucht ooc venckelzaet ende corinten in dese taerte doen ende de appelen in quartieren snijden ende dientse dan voorts.	To To make apple pie
\end{minipage}
\begin{minipage}{0.05\textwidth}
\ \ \ 
\end{minipage}
\begin{minipage}{.45\textwidth}
Make the dough as before. Chop your apples small or almost small. Put them in a disch and sprinkle well with sugar and powdered cinnamon with some ginger and some rosewater. Mix it well together with the apples. Put this mixture in your dough and in your pan. Add some butter here and there. Put your lid n it as before and bake it like a spinachpie. When it is baked enough sprinkle sugar and cinnamon. You can also add fennelseed and raisins
\end{minipage}

\medskip
\begin{minipage}{.45\textwidth}
127 Om een appeltaert te maken (De verstandige kok)
Neemt van de beste appelen, schiltse en snijtse in vierendeelen, de klockhuysen daeruyt gedaen. Koocktse met Rinse wijn in een aerden pot totdatse dick wordt. Doet'erby een goet deel suycker, gestooten kaneel, poeyer van sandelhout, rooswater. Wrijft het altesamen met een houte lepel door een omgekeerde teems. Doet'et in de korst en backtse dan in den oven. 't Sal goet zijn.	
\end{minipage}
\begin{minipage}{0.05\textwidth}
\ \ \ 
\end{minipage}
\begin{minipage}{.45\textwidth}
127 127 To make appeltart
Take  the best apples, peel and cut them in four parts. Without the cores. Boil them with Rhine wine in an earthen pot till it thickens. Add a good part sugar, powdered cinnamon, powder of sandalwood, rosewater.
Rub everything together with a wooden spoon through a ‘omgekeerde teems’ (sift?)
Put it in a crust and bake it in the oven. It shall be good
\end{minipage}

\medskip
\begin{minipage}{.45\textwidth}
[5] Om appel taert te maeken  (KANTL 15)
Neemt gember sofferaen vygen ende gruen blaeder van peterselien met zuycker ende kaneel	
\end{minipage}
\begin{minipage}{0.05\textwidth}
\ \ \ 
\end{minipage}
\begin{minipage}{.45\textwidth}
2.5. To make apple pie
Take ginger, saffron, figs and green leaves of parsley with sugar and cinnamon.
\end{minipage}

\bibliographystyle{apalike}
\bibliography{finneboonen-middleages}
\end{document}